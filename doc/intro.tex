\hspace{6.5cm}\begin{minipage}{10cm}\textbf{auger} \emph{\"o$^\prime$g\schwa r, n.} a carpenter's
boring tool\\(\emph{Chamber's Concise Dictionary})
\end{minipage}\\\\
In a forthcoming paper we analyzed unique regions in 18 mammalian
genomes. In this document we explain our analysis of unique genomic
regions by describing in full detail their detection and annotation in
one of one of these genomes, that of human. This is done in three
steps, pick unique regions, annotate unique regions with genes, and
carry out functional enrichment analysis of these genes.

To pick unique regions, a sliding window analysis of the focal genome
is carried out
using \ty{macle}\footnote{\ty{github.com/evolbioinf/macle}}. The
windows returned by a \ty{macle} run need to be merged into
non-overlapping intervals, which is done with the program \ty{merwin}
(Chapter~\ref{ch:me}). In the annotation step we link these intervals
with intersecting promoters. The corresponding genes get linked with
GO terms. So the annotation step requires a map between gene IDs and
GO terms. We store this map in a \ty{gene2go} file and show its
construction in Chapter~\ref{ch:g2g}. The final enrichment analysis
also relies on \ty{gene2go} files.
